\section{xmtrace - Trace Tool}

xmtrace is a tracing tool based on the rtl\_tracer included in RTLinux/GPL, written by Michael Barabanov, and has been reimplemented 
for the use in XtratuM. This manual only covers the use of xmtrace, detailed information on how xmtrace works internally can be 
found in \cite{xmtrace}. This section gives a step by step introduction on how to run the simple example, and after that the
code of the simple example is explained to allow you to use xmtrace in your application.\\

\subsection{Run an Example}\label{runex}
The first thing what we have to do, is to load the xmtrace\_shm.ko Linux kernel module. This module gets the address of the shared 
memory devices in use and passes the address to the part of the tracer that is located in the core. At the end of the initialization,
it starts the tracer. From this point in time the tracer is running, and tracing events can be retrieved.

\begin{verbatim}
$ insmod xmtrace/xmtrace_shm.ko
\end{verbatim}


\begin{verbatim}
$ cd xmtrace
$ ./tracer > data.raw
\end{verbatim}

switch to a second console

\begin{verbatim}
$ cd user_tools/examples/tracer
$ ../../xmloader/./load.xm tracer.xmd
\end{verbatim}

If you switch back to the first console now, you can stop ./tracer, and have a look at data.raw. 
In it you will see the output of the tracer that looks something like this:

\begin{verbatim}
....
sorry no trace ready at cpos 0 (data_ready) <00000000> (overwritten) 0 finalize 0
sorry no trace ready at cpos 0 (data_ready) <00000000> (overwritten) 0 finalize 0
sorry no trace ready at cpos 0 (data_ready) <00000000> (overwritten) 0 finalize 0
DUMP: cpos 0 (data_ready) <7fffffff> (overwritten) 70 finalize 101
D 0 -181177411 hw_save_flags                 0x82 <0xce8200cc>
D 0    165 hard cli                         0 <0xce8200a6>
D 0    177 hw_restore_flags              0x82 <0xce8200e9>
D 0  13435 hw_save_flags                 0x82 <0xce8200cc>
D 0    206 hw_save_flags                 0x96 <0xce8200cc>
D 0    327 hard cli                         0 <0xce8200a6>
D 0    178 hw_restore_flags              0x96 <0xce8200e9>
D 0  15054 hw_save_flags                 0x96 <0xce8200cc>
D 0    164 hard cli                         0 <0xce8200a6>
D 0   3257 hw_save_flags                 0x92 <0xce8200cc>
D 0    165 hard cli                         0 <0xce8200a6>
D 0   3801 hw_restore_flags              0x92 <0xce8200e9>
D 0    181 hw_restore_flags              0x96 <0xce8200e9>
D-1   4013 scheduler in                     0 <0xce82245b>
D-1    328 hw_save_flags                 0x82 <0xce8200cc>
D-1    183 hard cli                         0 <0xce8200a6>
D-1   5906 hw_save_flags                 0x82 <0xce8200cc>
D-1    440 hard cli                         0 <0xce8200a6>
D-1    714 hw_restore_flags              0x82 <0xce8200e9>
D-1    977 hw_save_flags                 0x92 <0xce8200cc>
D-1    163 hard cli                         0 <0xce8200a6>
D-1    393 hw_restore_flags              0x92 <0xce8200e9>
D-1    170 hw_save_flags                 0x96 <0xce8200cc>
D-1    368 hard cli                         0 <0xce8200a6>
D-1    455 hw_save_flags                 0x92 <0xce8200cc>
D-1    163 hard cli                         0 <0xce8200a6>
D-1   3762 hw_restore_flags              0x92 <0xce8200e9>
D-1    171 hw_restore_flags              0x96 <0xce8200e9>
D-1    173 scheduler in                     0 <0xce82245b>
D-1    162 hw_save_flags                 0x92 <0xce8200cc>
D-1    162 hard cli                         0 <0xce8200a6>
D-1    246 hw_save_flags                 0x96 <0xce8200cc>
D-1    163 hard cli                         0 <0xce8200a6>
D-1    627 hw_save_flags                 0x82 <0xce8200cc>
D-1    406 hard cli                         0 <0xce8200a6>
D-1    501 hw_restore_flags              0x82 <0xce8200e9>
D-1    177 hw_restore_flags              0x96 <0xce8200e9>
D-1    177 hw_restore_flags              0x92 <0xce8200e9>
D-1    164 scheduler out                  0x1 <0xce8225ed>
D 1   2920 xm-domain                   0x1267 <0xce8213fe>
That was trace # 1 (lost 0 traces befor this trace)
DUMP: cpos 1 (data_ready) <7ffffffe> (overwritten) 70 finalize 101
D 1 -181114183 hw_save_flags                 0x92 <0xce8200cc>
D 1    193 hard cli                         0 <0xce8200a6>
D 1    259 hw_save_flags                 0x86 <0xce8200cc>
D 1    354 hard cli                         0 <0xce8200a6>
D 1    489 hard sti                         0 <0xce820086>
D 1    303 hw_save_flags                0x296 <0xce8200cc>
D 1    169 hard cli                         0 <0xce8200a6>
D 1    541 hw_restore_flags             0x296 <0xce8200e9>
D 1    395 hard cli                         0 <0xce8200a6>
D 1    205 hw_restore_flags              0x86 <0xce8200e9>
D 1    200 hw_restore_flags              0x92 <0xce8200e9>
D 1    425 xm-domain                        0 <0xce8213fe>
That was trace # 2 (lost 0 traces befor this trace)
DUMP: cpos 2 (data_ready) <7ffffffc> (overwritten) 70 finalize 101
D 1 -181110149 xm-domain                      0x1 <0xce8213fe>
That was trace # 3 (lost 0 traces befor this trace)
DUMP: cpos 3 (data_ready) <7ffffff8> (overwritten) 70 finalize 101
D 1 -181109454 xm-domain                      0x2 <0xce8213fe>
....
\end{verbatim}

First you see some lines where no trace events have happend, followed by the traces.
Each trace starts with a line similart to this:

\begin{verbatim}
DUMP: cpos 2 (data_ready) <7ffffffc> (overwritten) 70 finalize 101
\end{verbatim}

and ends with a line like this:

\begin{verbatim}
That was trace # 3 (lost 0 traces befor this trace)
\end{verbatim}

Between these two lines, the trace events in the last trace are listed. Lets
see what a trace event looks like:

\begin{verbatim}
D-1    164 scheduler out                  0x1 <0xce8225ed>
\end{verbatim}

D-1 donates the which domain's context the trace event has happend. In our example it happend in the XtratuM kernel context,
therefore it is D-1. If you look at the trace, you will also find trace events that happened in domain 0 and  domain 1. \\
Next there is the relative time to the last trace event in nanoseconds (e.g. 164ns), followed by the name of the
trace event (e.g. scheduler out), an optional value that is passed to the tracer (e.g. 0x1 - the number of the domain
scheduled during this scheduler call, and the address of the tracepoint.\\

In the next step, we are going to resolve the addresses of the tracepoints into symbols. To do this, we first have
to create a system map, and then send the tracing data into the symbol resolve script.\\

\begin{verbatim}
$ cd xmtrace
$ ./createmap.sh
$ cat data.raw | ./addr2sym.sh -m MySystem.map > data.lst
\end{verbatim}

This creates a version of the trace from above, where the addresses are resovled into symbol names + offsets:
 
\begin{verbatim}
....
sorry no trace ready at cpos 0 (data_ready) <00000000> (overwritten) 0 finalize 0
sorry no trace ready at cpos 0 (data_ready) <00000000> (overwritten) 0 finalize 0
sorry no trace ready at cpos 0 (data_ready) <00000000> (overwritten) 0 finalize 0
DUMP: cpos 0 (data_ready) <7fffffff> (overwritten) 70 finalize 101
D 0 -181177411 hw_save_flags                 0x82 <hw_save_flags+0x1c 
D 0    165 hard cli                         0 <hw_cli+0x16 
D 0    177 hw_restore_flags              0x82 <hw_restore_flags+0x19 
D 0  13435 hw_save_flags                 0x82 <hw_save_flags+0x1c 
D 0    206 hw_save_flags                 0x96 <hw_save_flags+0x1c 
D 0    327 hard cli                         0 <hw_cli+0x16 
D 0    178 hw_restore_flags              0x96 <hw_restore_flags+0x19 
D 0  15054 hw_save_flags                 0x96 <hw_save_flags+0x1c 
D 0    164 hard cli                         0 <hw_cli+0x16 
D 0   3257 hw_save_flags                 0x92 <hw_save_flags+0x1c 
D 0    165 hard cli                         0 <hw_cli+0x16 
D 0   3801 hw_restore_flags              0x92 <hw_restore_flags+0x19 
D 0    181 hw_restore_flags              0x96 <hw_restore_flags+0x19 
D-1   4013 scheduler in                     0 <xm_sched+0x1b 
D-1    328 hw_save_flags                 0x82 <hw_save_flags+0x1c 
D-1    183 hard cli                         0 <hw_cli+0x16 
D-1   5906 hw_save_flags                 0x82 <hw_save_flags+0x1c 
D-1    440 hard cli                         0 <hw_cli+0x16 
D-1    714 hw_restore_flags              0x82 <hw_restore_flags+0x19 
D-1    977 hw_save_flags                 0x92 <hw_save_flags+0x1c 
D-1    163 hard cli                         0 <hw_cli+0x16 
D-1    393 hw_restore_flags              0x92 <hw_restore_flags+0x19 
D-1    170 hw_save_flags                 0x96 <hw_save_flags+0x1c 
D-1    368 hard cli                         0 <hw_cli+0x16 
D-1    455 hw_save_flags                 0x92 <hw_save_flags+0x1c 
D-1    163 hard cli                         0 <hw_cli+0x16 
D-1   3762 hw_restore_flags              0x92 <hw_restore_flags+0x19 
D-1    171 hw_restore_flags              0x96 <hw_restore_flags+0x19 
D-1    173 scheduler in                     0 <xm_sched+0x1b 
D-1    162 hw_save_flags                 0x92 <hw_save_flags+0x1c 
D-1    162 hard cli                         0 <hw_cli+0x16 
D-1    246 hw_save_flags                 0x96 <hw_save_flags+0x1c 
D-1    163 hard cli                         0 <hw_cli+0x16 
D-1    627 hw_save_flags                 0x82 <hw_save_flags+0x1c 
D-1    406 hard cli                         0 <hw_cli+0x16 
D-1    501 hw_restore_flags              0x82 <hw_restore_flags+0x19 
D-1    177 hw_restore_flags              0x96 <hw_restore_flags+0x19 
D-1    177 hw_restore_flags              0x92 <hw_restore_flags+0x19 
D-1    164 scheduler out                  0x1 <xm_sched+0x1ad 
D 1   2920 xm-domain                   0x1267 <system_call_handler_asm_0x82+0x26 
That was trace # 1 (lost 0 traces befor this trace)
DUMP: cpos 1 (data_ready) <7ffffffe> (overwritten) 70 finalize 101
D 1 -181114183 hw_save_flags                 0x92 <hw_save_flags+0x1c 
D 1    193 hard cli                         0 <hw_cli+0x16 
D 1    259 hw_save_flags                 0x86 <hw_save_flags+0x1c 
D 1    354 hard cli                         0 <hw_cli+0x16 
D 1    489 hard sti                         0 <hw_sti+0x16 
D 1    303 hw_save_flags                0x296 <hw_save_flags+0x1c 
D 1    169 hard cli                         0 <hw_cli+0x16 
D 1    541 hw_restore_flags             0x296 <hw_restore_flags+0x19 
D 1    395 hard cli                         0 <hw_cli+0x16 
D 1    205 hw_restore_flags              0x86 <hw_restore_flags+0x19 
D 1    200 hw_restore_flags              0x92 <hw_restore_flags+0x19 
D 1    425 xm-domain                        0 <system_call_handler_asm_0x82+0x26 
That was trace # 2 (lost 0 traces befor this trace)
DUMP: cpos 2 (data_ready) <7ffffffc> (overwritten) 70 finalize 101
D 1 -181110149 xm-domain                      0x1 <system_call_handler_asm_0x82+0x26 
That was trace # 3 (lost 0 traces befor this trace)
DUMP: cpos 3 (data_ready) <7ffffff8> (overwritten) 70 finalize 101
D 1 -181109454 xm-domain                      0x2 <system_call_handler_asm_0x82+0x26 
....
\end{verbatim}


\subsection{Using the Tracer}

This section analyzes the example that has been run in section \ref{runex}, and shows you how to
add tracing events into your application, and how to create new events and event-classes. This simple
example contains everything you need to trace your own applications.

\lstset{language=C}
\begin{lstlisting}
#include <xm_syscalls.h>

void handler (int event, void *a) {
  //write_scr ("irq 0\n", 6);
  write_scr("hola\n", 6);
     //pass_event (event);
     //unmask_event (0);
}

int kmain (void) {
  struct xmitimerval req = {{0, 200000}, {0, 0}};
  struct xmtimespec t;
  int i;

  install_event_handler (0, handler);
  
  set_timer (&req, 0);

  unmask_event (0);

  get_time(&t);

  xmtrace_user(20,4711);
  xmtrace_user(18,4711);

  xprint("nsec %lu sec %lu\n",t.tv_nsec, t.tv_sec);
  
  enable_events_flag();
  while (1){
	  for(i=0;i< 100; i++){
			  
  		xmtrace_user(20,i);  //RTL_TRACE_USER2
  		xmtrace_user(18,0);  //RTL_TRACE_FINALIZE
	  }
  suspend_domain (0,0);
  }
  return 0;
}


\end{lstlisting}
